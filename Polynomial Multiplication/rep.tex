\documentclass[a4paper,11pt]{article}
\usepackage[utf8]{inputenc}
\usepackage{times}

\usepackage{pgfplots}
\pgfplotsset{width=10cm}

\title{Report for assignment 8}
\author{Ankit Bansal (12EC30003)}

\begin{document}

\maketitle

\paragraph{Polynomial Multiplication}
\begin{enumerate}
 \item \textbf{Overview of the code}
The code take two frequencies f1 and f2 as inputs and then generate two sinusoids of that frequencies. FFT of these two sinusoids is generated and plotted using Latex. The FFT of the  two sinusoid is then shifted in frequency domain as per the problem statement and then added together. Inverse FFT of the combined signal is taken and plotted.

The program outputs text files to plot graphs using Latex.
\begin{description}
  \item[FFT:]
By far the most commonly used FFT is the Cooley–Tukey algorithm. This is a divide and conquer algorithm that recursively breaks down a DFT of any composite size N = N1N2 into many smaller DFTs of sizes N1 and N2, along with O(N) multiplications by complex roots of unity traditionally called twiddle factors.

Time Complexity :: \textbf{O$\left( N*log_2N \right)$}

  \item[iFFT]

Time Complexity :: \textbf{O$\left( N*log_2N \right)$}
\end{description}
\clearpage 
\end{enumerate}
\addtolength{\oddsidemargin}{-.875in}
% Preamble: \pgfplotsset{width=7cm,compat=1.11}

\begin{tikzpicture}[scale = 1.6]
\begin{axis}[
      title= Singnal 1;,
      xlabel={x axis},
      ylabel={yaxis},
      grid=major
]
\addplot[mark=none,red] table [x=x,y=y] {u1.txt};
\end{axis}
\end{tikzpicture}

\begin{tikzpicture}[scale = 1.6]
\begin{axis}[
      title=Signal 2,
      xlabel={x axis},
      ylabel={yaxis},
      grid=major
]
\addplot[mark=none,red] table [x=x,y=y] {u2.txt};
\end{axis}
\end{tikzpicture}

\begin{tikzpicture}[scale = 1.6]
\begin{axis}[
title=FFT Signal 1,
xlabel={x axis},
ylabel={yaxis},
grid=major
]
\addplot[mark=none,red] table [x=x,y=y] {fftu1.txt};
\end{axis}
\end{tikzpicture}

\begin{tikzpicture}[scale = 1.6]
\begin{axis}[
title=FFT Signal 2,
xlabel={x axis},
ylabel={yaxis},
grid=major
]
\addplot[mark=none,red] table [x=x,y=y] {fftu2.txt};
\end{axis}
\end{tikzpicture}

\begin{tikzpicture}[scale = 1.6]
\begin{axis}[
title=iFFT Signal 2,
xlabel={x axis},
ylabel={yaxis},
grid=major
]
\addplot[mark=none,red] table [x=x,y=y] {ifftu1.txt};
\end{axis}
\end{tikzpicture}

\begin{tikzpicture}[scale = 1.6]
\begin{axis}[
title=iFFT Signal 2,
xlabel={x axis},
ylabel={yaxis},
grid=major
]
\addplot[mark=none,red] table [x=x,y=y] {ifftu2.txt};
\end{axis}
\end{tikzpicture}


\end{document}
